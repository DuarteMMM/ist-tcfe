\section{Introduction} \label{sec:introduction}

In this laboratory assignment, a band pass filter (BPF) with an OP-AMP was implemented and studied. Using the circuit shown in Figure \ref{fig:CircuitDraw}, a theoretical analysis was made and its results were obtained by using the Octave math tool. Moreover, an Ngspice script was made in order to simulate this circuit. In both cases, the values of the central frequency, the input and output impedances at this frequency and the gain in the passband were determined. The plot of the frequency response has been obtained for both analyses, in which the values used for the resistances and capacitances shown in Figure \ref{fig:CircuitDraw} are presented in Table \ref{tab:chosen_values} of Section \ref{sec:analysis}. As seen below, designations have been assigned to each node in the circuit.
\par
As opposed to the previous laboratory assignments, it was also possible to test out this circuit presentially. Therefore, the circuit's gain was determined for different values of the components and these results will be shown in this report. It is worth noting that, even though the resistance $R_{2p}$ ended up not being considered for the theoretical and simulation analysis, it is still shown below, as it was used in the laboratory.

\begin{figure}[H] \centering
  \includegraphics[width=0.95\linewidth]{CircuitDraw.pdf}
  \caption{Circuit to be analysed in this laboratory assignment.}
  \label{fig:CircuitDraw}
\end{figure}
