\section{Introduction} \label{sec:introduction}

In this laboratory assignment, a bandpass filter (BPF) with an OP-AMP was implemented and studied. Using the circuit shown in Figure \ref{fig:CircuitDraw}, a theoretical analysis were made and its results were obtained by using the Octave math tool. Moreover, Ngspice scripts was made in order to simulate this circuit. In both cases, the values of the central frequency, the input and output impedances at this frequency and the gain in the passband were determined. The plots of the frequency response (gain and phase) have been obtained for both analyses. As seen below, designations have been assigned to each node in the circuit.
\par
As opposed to the previous assignments, it was also possible to test out this circuit in the laboratory and different configurations were used. Therefore, the circuit's gain was determined for different values of the resistances and these results will be shown in this report. It is worth noting that, even though the resistance $R_{2p}$, in parellel with $R_2$, ended up not being considered for the theoretical and simulation analyses, it is still shown below, as it was used in the laboratory.

\begin{figure}[H] \centering
  \includegraphics[width=0.95\linewidth]{CircuitDraw.pdf}
  \caption{Circuit to be analysed in this laboratory assignment}
  \label{fig:CircuitDraw}
\end{figure}
