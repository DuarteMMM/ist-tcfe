\section{Simulation Analysis} \label{sec:simulation}

The simulation utilized a given op-amp in order to simulate the circuit in Figure 1. 



The input and output impedances of the system as a whole were determined. The output impedance was simulated with $V_{in}$ turned off and a voltage source with amplitude of 1V in the output. The results obtained are shown in Table \ref{tab:impe}, together with the theoretical analysis'.

\begin{table}[H]
  \centering
  \begin{tabular}{cc}
    \begin{tabular}{|c|c|}
      \hline
      \multicolumn{2}{|c|}{\bf \color{blue} Theoretical} \\
      \hline
          {\bf Designation} & {\bf Value [$\Omega$]} \\ \hline
          \input{../mat/Impedances.tex}%suposto ser mat
    \end{tabular}
    \qquad
    \begin{tabular}{|c|c|}
      \hline
      \multicolumn{2}{|c|}{\bf \color{blue} Simulation} \\
      \hline    
          {\bf Designation} & {\bf Value [$\Omega$]} \\ \hline
          \input{../sim/impedances.tex}
    \end{tabular}
  \end{tabular}
  \caption{Comparison between theoretical and simulation's input and output impedances considering the whole circuit.}
  \label{tab:impe}
\end{table}

As it was intended, the input impedance is high and the output impedance is really small.
\par
Now, the plots of the output signal and the gain are presented. For a transient analysis of frequency f=1kHz for the the output has been plotted in Figure \ref{fig:out1}. We can notice that there are no observable losses from the original signal, that is, the wave is still sinusoidal. The plot was made ate a later time (not t=0) when there are no transient variances in the signal.

\begin{figure}[H]
  \begin{subfigure}{.49\linewidth}
    \centering
    \includegraphics[width=0.95\linewidth]{../sim/vo1.pdf}
    \footnotesize
  \caption{Output signal variance with time for f=1kHz}
   \label{fig:out1}
  \end{subfigure}
  \hspace{5mm}
  \begin{subfigure}{.49\linewidth}
    \centering
  \includegraphics[width=0.95\linewidth]{../sim/gain.pdf}
  \caption{Gain}
  \label{fig:out2}
  \end{subfigure}
\end{figure}

Regarding the plot of the gain, we can see that the passband effect is achieved being that the gain is only maximum for an intermediate level of frequencys.

 \begin{table}[H]
   \centering
  \begin{tabular}{cc}
    \begin{tabular}{|c|c|}
      \hline
      \multicolumn{2}{|c|}{\bf \color{blue} Theoretical} \\
      \hline
          {\bf Designation} & {\bf Value} \\ \hline
          \input{../mat/GainFrequencies.tex}
    \end{tabular}
    \qquad
    \begin{tabular}{|c|c|}
      \hline
      \multicolumn{2}{|c|}{\bf \color{blue} Simulation} \\
      \hline    
          {\bf Designation} & {\bf Value [Hz or dB]} \\ \hline
          \input{../sim/valsim_tab.tex}
    \end{tabular}
  \end{tabular}
  \caption{Values obtained for the lower and higher cutoff frequencies (in Hz) and the final gain (in dB), for both the simulation and the theoretical analysis; bandwidth (in Hz) for the simulation.}
  \label{tab:rip}
\end{table}


Finally, the total monetary cost and the merit $M$ of the circuit have been calculated and are shown below in Table \ref{tab:rip1}. These were determined by using the results obtained from Ngspice. The cost is given by $cost=cost\; of\; resistors\; +\; cost\; of\; capacitor\; +\; cost\; of\; transistores$, in which each 1$k\Omega$ in the resistances costs 1 monetary unit (MU), as well as each 1$\mu F$ in the capacitance; the cost of each transistor is 0.1 MU. On the other hand, the merit M is given by

\begin{equation} \label{eq:merit}
  M=\frac{1}{cost\times GainDeviation \times CentralFrequencydeviation} 
\end{equation}


\begin{table}[H]
  \centering
  \begin{tabular}{|c|c|}
    \hline
        {\bf Designation} & {\bf Value} \\ \hline
        \input{../sim/merit_tab.tex}
  \end{tabular}
  \caption{Cost and merit obtained for this circuit.} 
  \label{tab:rip1}
\end{table}

Cost and merit observations
