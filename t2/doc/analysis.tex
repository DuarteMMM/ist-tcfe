\section{Theoretical Analysis}
\label{sec:analysis}

\subsection{Exercise 1}

In this section, the circuit shown in Figure~\ref{fig:CircuitDraw} is analysed theoretically, by using the node method. The Kirchhoff Current Law (KCL) states that the sum of the currents converging or diverging in a node is null. The nodes considered for the following equations are those represented in Figure~\ref{fig:CircuitDraw}. Using KCL and Ohm's Law (which can also be written as $I=VG$) in nodes not connected to voltage sources and additional equations for nodes related by voltage sources, it is possible to obtain a linear system from which the voltages at nodes $V_1$ to $V_8$ and currents in resistances $R_1$ to $R_7$ can be determined.

\par
\vspace{1mm}

The following linear system is obtained:

\begin{equation}
  \begin{pmatrix}
    1 & 0 & 0 & 0 & 0 & 0 & 0 & 0 \\
    -\frac{1}{R_1} & \frac{1}{R_1}+\frac{1}{R_2}+\frac{1}{R_3} & -\frac{1}{R_2} & 0 & -\frac{1}{R_3} & 0 & 0 & 0 \\
    0 & -K_b-\frac{1}{R_2} & \frac{1}{R_2} & 0 & K_b & 0 & 0 & 0 \\
    0 & 0 & 0 & 1 & 0 & 0 & 0 & 0 \\
    0 & -\frac{1}{R_3} & 0 & -\frac{1}{R_4} & \frac{1}{R_3}+\frac{1}{R_4}+\frac{1}{R_5} & -\frac{1}{R_5} & -\frac{1}{R_7} & \frac{1}{R_7} \\
    0 & K_b & 0 & 0 & -\frac{1}{R_5}-K_b & \frac{1}{R_5} & 0 & 0 \\
    0 & 0 & 0 & 0 & 0 & 0 & \frac{1}{R_6}+\frac{1}{R_7} & -\frac{1}{R_7} \\
    0 & 0 & 0 & -\frac{K_d}{R_6} & 1 & 0 & \frac{K_d}{R_6} & -1
  \end{pmatrix}
  \begin{pmatrix}
    V_1  \\
    V_2  \\
    V_3  \\
    V_4  \\
    V_5  \\
    V_6  \\
    V_7  \\
    V_8  \\
  \end{pmatrix}
  =
  \begin{pmatrix}
    V_s \\
    0   \\
    0   \\
    0   \\
    0   \\
    0   \\
    0   \\
    0   \\
  \end{pmatrix}
  \label{eq:Exercise1LinearSystem}
\end{equation}

By solving the linear system~\ref{eq:Exercise1LinearSystem}, the following values for node voltages and branch currents (calculated by using Ohm's Law) are obtained:

\begin{table}[H]
  \centering
  \begin{tabular}{|c|c|}
    \hline
        {\bf Designation} & {\bf Value [A or V]} \\ \hline
        \input{Exercise1.tex}
  \end{tabular}
  \caption{Values of node voltages (in volts) and branch currents (in amperes). Current $I_i$ corresponds to the current passing through resistance $R_i$.}
  \label{tab:Exercise1Theoretical}
\end{table}

