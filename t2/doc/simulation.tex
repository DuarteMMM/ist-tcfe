\section{Simulation Analysis}
\label{sec:simulation}

\subsection{Exercise 1} \label{sec:Ex1Sim}

Similarly to Exercise 1 in the Theoretical Analysis, the voltages in all nodes and currents in all branches are obtained for $t<0$, situation in which it is considered that these values have been static for a long time. Thus, using Ngspice, an operating point analysis is made. In Table \ref{tab:Exercise1Both}, shown below, the simulated operating point results for the circuit presented in Figure \ref{fig:CircuitDraw} are shown alongside those obtained in the Theoretical Analysis. Again, currents designated below as $I_i$ refer to the currents passing through the respective resistances, $R_i$.

\vspace{-5mm}
\begin{table}[H]
  \centering
  \begin{tabular}{cc}
    \begin{tabular}{|c|c|}
      \hline
      \multicolumn{2}{|c|}{\bf \color{blue} Theoretical} \\
      \hline
          {\bf Designation} & {\bf Value [A or V]} \\ \hline
          \input{Exercise1.tex}
    \end{tabular}
    \qquad
    \begin{tabular}{|c|c|}
      \hline
      \multicolumn{2}{|c|}{\bf \color{blue} Simulation} \\
      \hline    
          {\bf Designation} & {\bf Value [A or V]} \\ \hline
          \input{../sim/Exercise1_tab.tex}
    \end{tabular}
  \end{tabular}
  \caption{Exercise 1 - comparison between theoretical analysis and operating point analysis's results. Currents are in amperes; voltages $V_i$ are in volts.}
  \label{tab:Exercise1Both}
\end{table}

Comparing the results, we can notice almost no differences. This was to be expected, since the circuit only contains linear components. There is only a small difference between the two values of $I_c$, although it is negligible. The value that should, in theory, be obtained is zero, because the voltages in the capacitor's terminals should be constant. The small value obtained with Octave must be due to computacional errors in approximations done while solving the linear system.


\subsection{Exercise 2} \label{sec:Ex2Sim}

Now, similarly to Exercise 2 in the Theoretical Analysis, we check how the system behaves when $v_s=0$ and the capacitor is replaced by a voltage souce $V_x=V(6)-V(8)$, where the voltages $V(6)$ and $V(8)$ are taken from the values obtained in Section \ref{sec:Ex1Sim}. This procedure must be done in order to determine the initial conditions for the voltages in nodes 6 and 8, $v_6(0)$ and $v_8(0)$, respectively, which will be used to obtain the natural and total responses of the circuit in the following sections.

\begin{table}[H]
  \centering
  \begin{tabular}{cc}
    \begin{tabular}{|c|c|}
      \hline
      \multicolumn{2}{|c|}{\bf \color{blue} Theoretical} \\
      \hline
          {\bf Designation} & {\bf Value [A or V]} \\ \hline
          $I_1$ & 1.181855E-03 \\ \hline 
$I_2$ & 1.237205E-03 \\ \hline 
$I_3$ & 5.535009E-05 \\ \hline 
$I_4$ & -2.553371E-04 \\ \hline 
$I_5$ & -1.048370E-03 \\ \hline 
$I_6$ & -1.437192E-03 \\ \hline 
$I_7$ & -1.437192E-03 \\ \hline 
$I_b$ & 1.237205E-03 \\ \hline 
$I_c$ & -2.285575E-03 \\ \hline 
$I_{V_s}$ & 1.181855E-03 \\ \hline 
$I_{V_d}$ & 8.483826E-04 \\ \hline 
$V_1$ & 0.000000E+00 \\ \hline 
$V_2$ & 1.229766E+00 \\ \hline 
$V_3$ & 3.706477E+00 \\ \hline 
$V_5$ & 1.060067E+00 \\ \hline 
$V_6$ & 4.240919E+00 \\ \hline 
$V_7$ & -2.955652E+00 \\ \hline 
$V_8$ & -4.401288E+00 \\ \hline 

    \end{tabular}
    \qquad
    \begin{tabular}{|c|c|}
      \hline
      \multicolumn{2}{|c|}{\bf \color{blue} Simulation} \\
      \hline    
          {\bf Designation} & {\bf Value [A or V]} \\ \hline
          \input{../sim/Exercise2_tab.tex}
    \end{tabular}

  \end{tabular}
  \caption{Exercise 2 - comparison between theoretical analysis and operating point analysis's results. Currents are in amperes; voltages $V_i$ are in volts.}
  \label{tab:Exercise2Both}
\end{table}

As seen above, there are no differences between the values obtained in the different analysis (even though the values might not be presented with the same number of decimal points), as expected, due to the linearity of the components. Having the values of the voltages in nodes 6 and 8, we may now determine the natural solution of the system for $t>0$, which is what will be done next.

\subsection{Exercise 3} \label{sec:Ex3Sim}

In this exercise, the natural response of the circuit (i.e., with $v_s=0$) is simulated by using Ngspice's transient analysis mode. The values of the voltages in nodes 6 and 8 obtained before are utilized as boundary conditions, using Ngspice's directive \texttt{.ic}. Figure \ref{fig:Ex3_Image} shows the result obtained for the voltage in node 6 in the time interval [0,20] ms.

\begin{figure}[H]
  \centering
  \small
  \includegraphics[width=0.5\linewidth]{../sim/ex3_image.pdf}
  \caption{Natural solution $v_{6n}(t)$ (in volts) in the time interval [0,20] ms.}
  \label{fig:Ex3_Image}
\end{figure}

As we can see, the solution is incredibly similar to the one obtained in the Theoretical Analysis, i.e, an exponential, with the initial value given by $v_6(0)$ determined by replacing the capacitor with the voltage source $V_x$, that decays with time at an apparently equal rate.


\subsection{Exercise 4} \label{sec:Ex4Sim}

Having obtained the inicial conditions, we now consider the voltage given in Figure \ref{fig:CircuitDraw} for $t>0$, $v_s(t)=sin(2\pi ft)$, in order to obtain the total response on node 6 (natural solution + forced solution) by using Ngspice. In Figure \ref{fig:Ex4_Image}, the voltages in two nodes are represented: the voltage v(6) in node 6, i.e., the response, and the voltage in node 1, v(1), which corresponds to the stimulus, because the voltage source connects node 1 to GND, as seen in Figure \ref{fig:CircuitDraw}. These are both plotted below for $f=1$ kHz.

\begin{figure}[H]
  \centering
  \includegraphics[width=0.6\linewidth]{../sim/ex4_image.pdf}
  \caption{Transient analysis - total response on node 6, $v_6(t)$, and stimulus, $v_s(t)=v_1(t)$, both in volts, in time interval [0,20] ms and for $f=1$ kHz.}
  \label{fig:Ex4_Image}
\end{figure}


As expected, the plot of the total response corresponds to the sum of an exponential and a sinusoidal function. Both graphs look exactly the same as the one obtained in the theoretical results.

\subsection{Exercise 5} \label{sec:Ex5Sim}

In this last section, the frequency analysis was performed on the circuit for frequencies ranging from $f=0.1$ Hz to $f=1$ MHz. In Figure (\ref{fig:Ex5_Image_Magnitude}), the magnitude responses for $v_s$, $v_6$ and $v_c$ are plotted in dB and, in Figure (\ref{fig:Ex5_Image_Phase}), the respective phases in degrees are all represented. Both graphs use a frequency logscale.

\begin{figure}[H]
  \begin{subfigure}{.49\linewidth}
    \centering
    % include first image
    \includegraphics[width=1.\linewidth]{../sim/ex5_image_magnitude.pdf}
    \footnotesize
    \caption{Frequency analysis - magnitudes of $v_s(f)=v_1(f)$, $v_6(f)$ and $v_c(f)=v_6(f)-v_8(f)$ in frequency interval [$0.1$,$10^{6}$] Hz and in dB.}
    \label{fig:Ex5_Image_Magnitude}
  \end{subfigure}
  \hspace{5mm}
  \begin{subfigure}{.49\linewidth}
    \centering
    % include second image
    \includegraphics[width=1.\linewidth]{../sim/ex5_image_phase.pdf}  
    \caption{Frequency analysis - phases of $v_s(f)$, $v_6(f)$ and $v_c(f)=v_6(f)-v_8(f)$ in frequency interval [$0.1$,$10^{6}$] Hz and in degrees.}
    \label{fig:Ex5_Image_Phase}
  \end{subfigure}
\end{figure}

Having obtained graphs incredibly similar to those obtained in Section \ref{sec:Ex6Theo} of the Theoretical Analysis, the conclusions regarding these solutions are the same. Firstly, in Figure (\ref{fig:Ex5_Image_Magnitude}), we can verify what was learnt in class, since $v_c$ is inversely proportional to the frequency and so, when $f$ increases, $v_c$  tends to zero, thus it tends to $-\infty$ in dB. The phase is also explained by the limit of the $-atan$ when $f$ tends to $\infty$, as discussed before - it tends to $-90^{o}$.  As expected, the magnitude and phase of $v_s(=v_1)$ are constant. In contrast, those of $v_6$ change and depend on the frequency.
\par
For lower frequencies, meaning high impedance on the capacitor, it behaves similarly to a DC circuit, which results in almost constant amplitudes and phases. When $\omega^{2}R_{eq}^{2}C^{2}$ gets bigger, we observe a shift in behaviour, as we start noticing $v_6$'s dependence on the frequency. When we aproach high frequencies, meaning low impeadence in the capacitor, it starts behaving as a short circuit and so the amplitude of $v_6$ should tend to the amplitude of $v_8$ and its phase tends to $-180^{o}$.   
