\section{Simulation Analysis}
\label{sec:simulation}

\subsection{Operating Point Analysis}

Table~\ref{tab:op} shows the simulated operating point results for the circuit presented in Figure \ref{fig:CircuitDraw}. Voltages v$(j)$ correspond to the respective voltages $V_j$ in Table \ref{tab:Theoretical}. Besides this, currents designated below as i$j$ refer to the currents passing through the respective resistances, $R_j$.
%Therefore, @r1[i] is the value of $I_a$, @r2[i] is the value of $I_b$ and @r6[i]=@r7[i] is the value of $I_c$. On the other hand, similarly to what is indicated in Table \ref{tab:Theoretical}, v(1)=$V_c$ and v(4)-v(1)=$V_b$.

%\par

%It is also worth mentioning that @id[current] is the value of $I_d$ and that @gb[i] is the current $I_b$.

\begin{table}[H]
  \centering
  \begin{tabular}{|c|c|}
    \hline    
    {\bf Designation} & {\bf Value [A or V]} \\ \hline
    \input{../sim/op_tab.tex}
  \end{tabular}
  \caption{Operating point analysis table. Currents i$j$ are in amperes; voltages v$(j)$ are in volts.}
  \label{tab:op}
\end{table}

Comparing the theoretical analysis results presented in Table \ref{tab:Theoretical} and the results in Table \ref{tab:op}, we can notice no observable differences. This was to be expected, since the circuit has no time dependency - meaning it's equal at any point in time - and all the components are linear.
