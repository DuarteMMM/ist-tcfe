\section{Simulation Analysis}
\label{sec:simulation}

\subsection{Operating Point Analysis}

Table~\ref{tab:op} shows the simulated operating point results for the circuit presented in Figure \ref{fig:rc}

\begin{table}[H]
  \centering
  \begin{tabular}{|l|r|}
    \hline    
    {\bf Name} & {\bf Value [A or V]} \\ \hline
    \input{../sim/op_tab.tex}
  \end{tabular}
  \caption{Operating point table. A variable preceded by @ is a {\em current} in the unit Ampere; other variables are {\it voltages} in the unit Volt.}
  \label{tab:op}
\end{table}

Comparing the theoretical analysis results presented in table ?To-be-determined? and the results in table ~\ref{tab:op} we can see no observable differences. This is to be expected since the circuit has no time dependency meaning its equal at any point in time and all the components are linear. For all these motives if we compare both tables all values match and there are no reasons why theys should not.
