\section{Introduction}
\label{sec:introduction}

% state the learning objective
The objective of this laboratory assignment is to analyse the circuit shown in Figure~\ref{fig:CircuitDraw}, that is, to find all node voltages and branch currents. In Figure~\ref{fig:CircuitDraw}, the nodes have been numbered, current names and directions have been assigned to all branches and potential $0$ has been assigned to one of the nodes.
\par
By running the Python script \texttt{t1\_datagen.py}, the values of the resistances $R_1$ to $R_7$, voltage $V_a$, current $I_d$ and constants $K_b$ and $K_c$ are obtained. These are shown in Table \ref{tab:GivenValues}, shown below. In Section~\ref{sec:analysis}, a theoretical analysis of the circuit and the results obtained with the Octave math tool are presented. In Section~\ref{sec:simulation}, the results obtained using the Ngspice simulation tool are shown. The conclusions of this study are outlined in Section~\ref{sec:conclusion}, in which the theoretical results obtained in Section~\ref{sec:analysis} are compared to those presented in Section~\ref{sec:simulation}.

\begin{figure}[H] \centering
  \includegraphics[width=0.8\linewidth]{CircuitDraw.pdf}
  \caption{Circuit to be analysed in this laboratory assignment.}
  \label{fig:CircuitDraw}
\end{figure}

\begin{table}[H]
  \centering
  \begin{tabular}{|c|c|c|c|}
    \hline
        {\bf Designation} & {\bf Value}  \\
        \hline
        $R_1$ & 1.04053890347 k$\Omega$ \\
        \hline
        $R_2$ & 2.00185929606 k$\Omega$ \\
        \hline
        $R_3$ & 3.06593231919 k$\Omega$ \\
        \hline
        $R_4$ & 4.15163583349 k$\Omega$ \\
        \hline
        $R_5$ & 3.03409481751 k$\Omega$ \\
        \hline
        $R_6$ & 2.05654586148 k$\Omega$ \\
        \hline
        $R_7$ & 1.00587575204 k$\Omega$ \\
        \hline
        $V_a$ & 5.16821048288 V\\
        \hline
        $I_d$ & 1.0127707267 A\\
        \hline
        $K_b$ & 7.29055867767 mS\\
        \hline
        $K_c$ & 8.22649929708 k$\Omega$ \\
        \hline
  \end{tabular}
  \caption[]{Values obtained by running the file \texttt{t1\_datagen.py}.}
  \label{tab:GivenValues}
\end{table}

