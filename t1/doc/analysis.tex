\section{Theoretical Analysis}
\label{sec:analysis}

In this section, the circuit shown in Figure~\ref{fig:rc} is analysed theoretically, determinig the values of voltage and current acording to the mesh method and the node mehtod.

\subsection{Mesh Method}

The mesh method is based in the KVL, stating that the current in a elementary mesh is some fixed value. Furthermore, we assume that the current going throw a component is the sum of the currents of the elementary meshes, or  elementary mesh, where the componet is. Then we apply the KVL law where, instead of using the Voltages, we use the current times the resistance because of the Ohm, which states that: 
\begin{equation}
  R=\frac{V}{I} \Leftrightarrow V=R \times I
  \label{eq:kvl}
\end{equation}

Then, we write the equations acording to this format, where R is the resistance of the component, and I is the sum of the currents applied to the resistance:

\begin{equation}
  \Sigma_iR_iI_i=0
  \label{eq:kvl}
\end{equation}

It's clear that we can only write the equations for the meshes which don't have some voltage source. In order to avoid errors, one shoul previously define the direction of the current going throw each component, as well as the meshes currents directions. If this equations are not enough to solve the problem we write the KVL, or in the most unlikely of possibilities the KCL.
\par
KVL states that the sum of the voltages in a mesh is zero, while the KCL states that the sum of the currents going to some point of the circuit is the same as the sum of the currents going out of that same point. Notice that KVL is valid both for elementary and not elementary knots.
\par
In order to make each step clear, first the whole system will be shown,with eachof the currents marked, then the equation for each mesh, and then, finally, the matrix to solve.
\par
Now, we are going to show the figure of the circuit with the currents in each mesh marked.
\begin{figure}[H] \centering
\includegraphics[width=0.8\linewidth]{TotalMesh.pdf}
\caption{Circuit with the currents marked}
\label{TotalMesh}
\end{figure}

For the mesh d it is clear that the correpondent current is the current of the current source. Therefore we have:
\begin{equation}
  Id=I_d
  \label{meshdfinal}
\end{equation}


For the mesh a, as it has a voltage source, we have to apply the KVL, obtaining:
\begin{equation}
  -Va+R_1Ia+R_3(Ia+Ib)+R4(Ia+Ic)=0
  \label{meshaKVL}
\end{equation}
Then we get the final mesh a equation as:
\begin{equation}
  Ia(R_1+R_3+R_4)+Ib(R_3)+Ic(R_4)=V_a
  \label{meshaKVLfinal}
\end{equation}



For the mesh b, we could simply write down:
\begin{equation}
  Ib=K_bV_b=K_b(R_3(Ia+Ib))
  \label{meshbKVL}
\end{equation}
Then we get the final mesh equation for b as:
\begin{equation}
  IaK_bR_3+Ib(K_bR_3-1)=0
  \label{meshbKVLfinal}
\end{equation}


For the mesh c we apply, again, the KVL, obtaining:
\begin{equation}
  -V_c+R_4(Ia+Ic)+Ic(R_6+R_7)=0
  \label{meshcKVL}
\end{equation}
One should substitute the $V_C$ for $K_cIc$, obtainig the final equation as:
\begin{equation}
  Ia(R_4)+Ic(R_4+R_6+R_7-K_c)=0
  \label{meshcKVL}
\end{equation}


Finally, we only have to join the four final equations, obtaining the system:
\par
\begin{equation}
\begin{pmatrix}
  R_1+R_3+R_4 & R_3 & R_4 \\
  K_bR_3 & K_bR_3-1 & 0 \\
  R_4 & 0 & R_4+R_6+R_7-K_c
\end{pmatrix}
\begin{pmatrix}
  Ia  \\
  Ib  \\
  Ic
\end{pmatrix}
=
\begin{pmatrix}
  V_a  \\
  0  \\
  0
\end{pmatrix}
  \end{equation}

\subsection{Node Method}
The node method is based in the KCL and it is based in the fact that a node have a determined value of voltage, if we define one as the ground. Basically we say that the voltage in a component is the voltage drop in the direction of the current. Imagining we are looking at the node z. Then the node method simply states that:
\begin{equation}
  \Sigma(Vz-Vi)Gi
\end{equation}
where i states for the node which connects with z throw a resistance. One notices that if there are nodes which are not connect throw resistances. Therefore, we use the original form of the KCL, in order to keep the differences in mind.
\par
In the figure showed above the ground is already marked, so one can simply name the nodes and start writing the equations. The nodes were labeled as  below:
\begin{figure}[H] \centering
\includegraphics[width=0.8\linewidth]{TotalNodes.pdf}
\caption{Circuit with the nodes marked}
\label{TotalNodes}
\end{figure}

The value of the node V0 is 0V, because it is connected to the ground.
We will simply ignore this equation.

For the node V2 we have:
\begin{equation}
V1(\frac{1}{R_4})+V2(-\frac{1}{R_4}-\frac{1}{R_6})+V4(\frac{1}{R_1})+V5(\frac{1}{R_6})+V6(-\frac{1}{R_1})=0  
\end{equation}

For the node V3 we have:
\begin{equation}
  V1(-\frac{1}{R_5}-K_b)+V3(\frac{1}{R_5})+V4(K_b)=I_d
  \end{equation}


For the node V4 we have:
\begin{equation}
V1(-\frac{1}{R_3})+V4(\frac{1}{R_1}+\frac{1}{R_2}+\frac{1}{R_3})+V6(-\frac{1}{R_1})+V7(-\frac{1}{R_2})=0  
\end{equation}

For the node V5 we have:
\begin{equation}
  V2(-\frac{1}{R_6})+V5(\frac{1}{R_6}+\frac{1}{R_7})=0
\end{equation}

For the node V7 we have:
\begin{equation}
  V1(K_b)+V4(-\frac{1}{R_2}-K_b)+V7(\frac{1}{R_2})=0
  \end{equation}

For the source $V_a$ we have:
\begin{equation}
  V6-V2=V_a
\end{equation}

For the source $V_c$ we have:
\begin{equation}
  V1+V5\frac{K_c}{R_6}+V2(-\frac{K_c}{R_6})=0
  \end{equation}

Then, joining all the equations above, one gets the following matricial equation:
\begin{equation}
\begin{pmatrix}
  -\frac{1}{R_3} & 0 & 0 & \frac{1}{R_1}+\frac{1}{R_3}+\frac{1}{R_2} & 0 & -\frac{1}{R1} & -\frac{1}{R_2} \\
  0 & -\frac{1}{R_6} & 0 & 0 & \frac{1}{R_6}+\frac{1}{R_7} & 0 & 0 \\
  0 & -1 & 0 & 0 & 0 & 1 & 0 \\
  1 & -\frac{K_c}{R_6} & 0 & 0 & \frac{K_c}{R_6} & 0 & 0 \\
  -\frac{1}{R_5}-K_b & 0 & \frac{1}{R_5} & K_b & 0 & 0 & 0 \\
  K_b & 0 & 0 & -\frac{1}{R_2}-K_b & 0 & 0 & \frac{1}{R_2} \\
  \frac{1}{R_4} & -\frac{1}{R_4}-\frac{1}{R_6} & 0 & \frac{1}{R_1} & \frac{1}{R_6} & -\frac{1}{R_1} & 0
\end{pmatrix}
\begin{pmatrix}
  V1  \\
  V2  \\
  V3  \\
  V4  \\
  V5  \\
  V6  \\
  V7  \\
\end{pmatrix}
=
\begin{pmatrix}
  0   \\
  0   \\
  V_a  \\
  0    \\
  I_d  \\
  0    \\
  0    \\
\end{pmatrix}
  \end{equation}


\subsection{Results}
Solving the matricial eqautions above we get the values below:
\begin{table}[H]
  \centering
  \begin{tabular}{|l|r|}
    \hline
     {\bf Name} & {\bf Value [A or V]} \\ \hline
    \input{ofile.tex}
  \end{tabular}
  \caption{Values of the variables calculated, currents in A and voltages in V}
  \label{tab1}
\end{table}



