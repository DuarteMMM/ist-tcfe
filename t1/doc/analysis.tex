\section{Theoretical Analysis}
\label{sec:analysis}

In this section, the circuit shown in Figure~\ref{fig:CircuitDraw} is analysed theoretically, by determining the values of voltages and currents, using the mesh and node methods.

\subsection{Mesh Method}

The Kirchhoff Voltage Law (KVL) states that the sum of voltages in a circuit loop is zero. In Figure~\ref{fig:TotalMesh}, the chosen directions for the mesh currents $I_a$, $I_b$, $I_c$ and $I_d$ are shown. It is also worth mentioning Ohm's Law, which states the following:
\par

\begin{equation}
  R=\frac{V}{I} \Leftrightarrow V=R \times I
  \label{eq:OhmLaw}
\end{equation}

Using this information, an equation is written for each of the four meshes represented in Figure~\ref{fig:TotalMesh}. The value of current $I_d$ is known and it was obtained by running \texttt{t1\_datagen.py}. By solving the system that comes from applying KVL, the values of $I_a$, $I_b$ and $I_c$ will be obtained. Knowing all the mesh currents and using the other values given by \texttt{t1\_datagen.py}, it is possible to know all node voltages and branch currents, using Ohm's Law.

\par
\vspace{1mm}

By applying KVL and Ohm's Law in mesh 'a', the following equation is obtained:

\begin{equation}
  -V_a+R_1I_a+R_3(I_a+I_b)+R_4(I_a+I_c)=0 \Leftrightarrow I_a(R_1+R_3+R_4)+I_b(R_3)+I_c(R_4)=V_a
  \label{eq:Mesh_a}
\end{equation}

For mesh 'b', we can write down:

\begin{equation}
  I_b=K_bV_b=K_b(R_3(I_a+I_b)) \Leftrightarrow I_a(K_bR_3)+I_b(K_bR_3-1)=0
  \label{eq:Mesh_b}
\end{equation}

The equation for mesh 'c' is:

\begin{equation}
  -V_c+R_4(I_a+I_c)+I_c(R_6+R_7)=0 \Leftrightarrow I_a(R_4)+I_c(R_4+R_6+R_7-K_c)=0
  \label{eq:Mesh_c}
\end{equation}

For mesh 'd', it is clear that the correpondant current is the same as the current source's. Therefore, by inspection, we simply obtain:

\begin{equation}
  I_d=I_d
  \label{eq:Mesh_d}
\end{equation}

Finally, by joining equations~\ref{eq:Mesh_a},~\ref{eq:Mesh_b} and~\ref{eq:Mesh_c} together, we get the following system:

\begin{equation}
  \begin{pmatrix}
    R_1+R_3+R_4 & R_3 & R_4 \\
    K_bR_3 & K_bR_3-1 & 0 \\
    R_4 & 0 & R_4+R_6+R_7-K_c
  \end{pmatrix}
  \begin{pmatrix}
    I_a  \\
    I_b  \\
    I_c
  \end{pmatrix}
  =
  \begin{pmatrix}
    V_a  \\
    0  \\
    0
  \end{pmatrix}
  \label{eq:LinearSystem1}
\end{equation}

\begin{figure}[H] \centering
  \includegraphics[width=0.8\linewidth]{TotalMesh.pdf}
  \caption{Circuit with the respective mesh currents.}
  \label{fig:TotalMesh}
\end{figure}


\subsection{Node Method}

The Kirchhoff Current Law (KCL) states that the sum of the currents converging or diverging in a node is null. The nodes considered for the following equations are those represented in Figure~\ref{fig:CircuitDraw}. Using KCL and Ohm's Law (which can also be written as $I=VG$) in nodes not connected to voltage sources and additional equations for nodes related by voltage sources, it is possible to obtain a linear system from which the voltages at nodes $V_1$ to $V_7$ can be determined. Using these values and equation~\ref{eq:OhmLaw}, currents $I_a$, $I_b$ and $I_c$ could be easily calculated.

\par
\vspace{1mm}

Let us now consider, in the first place, nodes not connected to voltage sources.

\par

For node $V_2$, we have that:
\begin{equation}
  V_1(G_4)+V_2(-G_4-G_6)+V_4(G_1)+V_5(G_6)+V_6(-G_1)=0
  \label{eq:Nodes_1}
\end{equation}

The equation for node $V_3$ is the following:
\begin{equation}
  V_1(-G_5-K_b)+V_3(G_5)+V_4(K_b)=I_d
  \label{eq:Nodes_2}
\end{equation}

For node $V_4$:
\begin{equation}
  V_1(-G_3)+V_4(G_1+G_2+G_3)+V_6(-G_1)+V_7(-G_2)=0
  \label{eq:Nodes_3}
\end{equation}

The equation for node $V_5$ is:
\begin{equation}
  V_2(-G_6)+V_5(G_6+G_7)=0
  \label{eq:Nodes_4}
\end{equation}

And finally, for node $V_7$, we have that:
\begin{equation}
  V_1(K_b)+V_4(-G_2-K_b)+V_7(G_2)=0
  \label{eq:Nodes_5}
\end{equation}

Now, additional equations can be obtained for the nodes related by voltage sources ($V_0$, $V_1$, $V_2$ and $V_6$).
\par
The value of the voltage in node $V_0$ is $0 V$, because it is connected to GND. Therefore, this equation will not be used in the final matrix.
\par
The source $V_a$ can be used to obtain the following equation:
\begin{equation}
  V_6-V_2=V_a
  \label{eq:Nodes_6}
\end{equation}

Finally, for node $V_1$, we have that $V_1$ - $V_0$ = $V_1$ = $V_c$ = $K_c$ $I_c$. Using Ohm's Law, we can replace $I_c$ by $(V_2-V_5)G_6$. Therefore, the equation obtained is:
\begin{equation}
  V_1+V_5(K_cG_6)+V_2(-K_cG_6)=0
  \label{eq:Nodes_7}
\end{equation}

Considering equations~\ref{eq:Nodes_1}-\ref{eq:Nodes_7} above, one gets the following linear system:

\begin{equation}
  \begin{pmatrix}
    1 & -\frac{K_c}{R_6} & 0 & 0 & \frac{K_c}{R_6} & 0 & 0 \\
    \frac{1}{R_4} & -\frac{1}{R_4}-\frac{1}{R_6} & 0 & \frac{1}{R_1} & \frac{1}{R_6} & -\frac{1}{R_1} & 0 \\
    -\frac{1}{R_5}-K_b & 0 & \frac{1}{R_5} & K_b & 0 & 0 & 0 \\
    -\frac{1}{R_3} & 0 & 0 & \frac{1}{R_1}+\frac{1}{R_3}+\frac{1}{R_2} & 0 & -\frac{1}{R1} & -\frac{1}{R_2} \\
    0 & -\frac{1}{R_6} & 0 & 0 & \frac{1}{R_6}+\frac{1}{R_7} & 0 & 0 \\
    0 & -1 & 0 & 0 & 0 & 1 & 0 \\
    K_b & 0 & 0 & -\frac{1}{R_2}-K_b & 0 & 0 & \frac{1}{R_2}
  \end{pmatrix}
  \begin{pmatrix}
    V_1  \\
    V_2  \\
    V_3  \\
    V_4  \\
    V_5  \\
    V_6  \\
    V_7  \\
  \end{pmatrix}
  =
  \begin{pmatrix}
    0   \\
    0   \\
    I_d \\
    0   \\
    0   \\
    V_a \\
    0   \\
  \end{pmatrix}
  \label{eq:LinearSystem2}
\end{equation}


\subsection{Results}
By solving the linear systems~\ref{eq:LinearSystem1} and~\ref{eq:LinearSystem2}, the following values for node voltages, $V_b$ and branch currents (calculated using $I_a$, $I_b$ and $I_c$) are obtained:

\begin{table}[H]
  \centering
  \begin{tabular}{|c|c|}
    \hline
        {\bf Designation} & {\bf Value [A or V]} \\ \hline
        \input{ofile.tex}
  \end{tabular}
  \caption{Values of voltages (in volts) and mesh currents (in amperes). Current $I_i$ corresponds to the current passing through resistance $R_i$.}
  \label{tab:Theoretical}
\end{table}



