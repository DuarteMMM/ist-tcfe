\section{Simulation Analysis} \label{sec:simulation}

In order to simulate this circuit with Ngspice, real models of npn and pnp transistors where used. We realised that this models are very complex in reagards to the thheoretical models used previously. Because of this we expect there to be significant difference between the results.
\par
The simulation was done by taking the base script and tinkering with the values in order to improve the model. We strived to improve the bandwidth mainly since this was, observing the output graphs and value, the component that was lacking the most given that this was a audio amplifier and should worrk for a wide range of frequencies.

We start by an inicial operating point analysis.

\begin{table}[H]
  \centering
  \begin{tabular}{|c|c|}
    \hline
        {\bf Designation} & {\bf Value [V]} \\ \hline
        \input{../sim/op_tab.tex}
  \end{tabular}
  \caption{Operating point analysis} 
  \label{tab:rip}
\end{table}

As we can see  the values correspond to the transistors being in a forward active reagion. This means for the NPN, that $V_{CE}>V_{BE}$, and that for de PNP $V_{EC}>V_{EB}$. As we observe the values of the theoretical and simulation analysis are not to far off eachother and that is to be expected since although the models used for the operating point analysis are good approximations they are still aproximations not the real model.
\par



Afterwards the input and output impeadances of the system as a hole (gain stage + output stage) were determined. The output impeadance was simulated with $V_{in}$ turned off and a voltage source with amplitude of 1 V in the output. The results obtained are the following in table \ref{tab:impe}.

\begin{table}[H]
  \centering
  \begin{tabular}{|c|c|}
    \hline
        {\bf Designation} & {\bf Value [$\Omega$]} \\ \hline
        \input{../sim/impedances.tex}
  \end{tabular}
  \caption{Output and Input impeadances} 
  \label{tab:impe}
\end{table}

As we wanted the input impeadance is high and the output is small. Although improvements where made to take the out impeadance as low as we could this value is, perhaps, still a bit high and if it was better maybe it would make a better amplifier. \par
We can now analise the graphics of the output signal and the gain.

\begin{figure}[H]
  \begin{subfigure}{.49\linewidth}
    \centering
    \includegraphics[width=0.95\linewidth]{../sim/vo2f.pdf}
    \footnotesize
  \caption{Output signal response to frequency}
   \label{fig:out1}
  \end{subfigure}
  \hspace{5mm}
  \begin{subfigure}{.49\linewidth}
    \centering
  \includegraphics[width=0.95\linewidth]{../sim/gain.pdf}
  \caption{Gain}
  \label{fig:out2}
  \end{subfigure}
\end{figure}

We can also see in a transient analysis for f=1kHz, how is the output of the signal in regards of time, as presented in figure \ref{fig:outtime}. It is observable that there are no losses of the original signal (still sinusoidal wave)  The plot after the inicial seconds in order to represent the system without transitory interferences.
 \begin{figure}[H]
    \centering
  \includegraphics[width=0.45\linewidth]{../sim/vo2.pdf}
  \caption{Output signal variance with time for f=1kHz}
  \label{fig:outtime}
\end{figure}

 

Analysing now the graphs obtained we are able to determine all the important information about the output. Looking initially at figure \ref{fig:out1}, we start by obtaining the lower(lowfrequence) and higher(highfrequence) cutoff frequencies which are determined by taking the maximum output signal and seeing where this drops 3 dB. With this two values we subtract them and obtained bandwidth. Finaly looking now at figure \ref{fig:out2} we obtain the maximum value of the gain. All the results are presented in table \ref{tab:rip}
\begin{table}[H]
  \centering
  \begin{tabular}{|c|c|}
    \hline
        {\bf Designation} & {\bf Value [Hz and dB]} \\ \hline
        \input{../sim/valsim_tab.tex}
  \end{tabular}
  \caption{Lower frequency(Hz), high frequency(Hz), bandwidth(Hz) and gain(dB)} 
  \label{tab:rip}
\end{table}



Finally, the total monetary cost and the merit $M$ of the circuit used have been calculated and are shown below in Table \ref{tab:rip1}. These were determined by using the results obtained from Ngspice. The cost is given by $cost=cost\; of\; resistors\; +\; cost\; of\; capacitor\; +\; cost\; of\; transistores$, in which each 1$k\Omega$ in the resistances costs 1 monetary unit (MU), as well as each 1$\mu F$ in the capacitance; the cost of each transistor is 0.1 MU. On the other hand, the  merit M is given by

\begin{equation} \label{eq:merit}
  M=\frac{voltageGain \times bandwidth}{cost\times lowerCutoffFreq} 
\end{equation}


\begin{table}[H]
  \centering
  \begin{tabular}{|c|c|}
    \hline
        {\bf Designation} & {\bf Value} \\ \hline
        \input{../sim/merit_tab.tex}
  \end{tabular}
  \caption{Cost and merit obtained for this circuit.} 
  \label{tab:rip1}
\end{table}
