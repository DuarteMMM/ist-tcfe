\section{Theoretical Analysis} \label{sec:analysis}

The values used for the circuit's resistances and capacitances were determined while doing the Simulation Analysis, in order to get the best possible results, whilst mantaining the monetary cost as low as possible and the merit as high as possible. Using these values, which will be further discussed in Section \ref{sec:simulation}, the theoretical model is applied in this Section, in order to verify its validity and the disparities with the results obtained from Ngspice. The resistances and capacitances used were the following:


\begin{table}[H]
  \centering
  \begin{tabular}{|c|c|}
    \hline    
    {\bf Name} & {\bf Value} \\ \hline
    \input{../mat/ChosenValues.tex}
  \end{tabular}
  \caption{Values used for resistances and capacitances in the circuit.}
  \label{tab:chosen_values}
\end{table}

As it was learnt in class, the Gain Stage can be simplified into the circuit shown below, by using the Thévenin's equivalent of the bias circuit.

\begin{figure}[H] \centering
  \includegraphics[width=0.8\linewidth]{CircuitDrawGainStage.pdf}
  \caption{Gain Stage circuit.}
  \label{fig:CircuitDrawGainStage}
\end{figure}

While doing an Operating Point Analysis of the circuit shown in Figure \ref{fig:CircuitDrawGainStage}, the capacitor becomes an open-circuit, thus only the resistor $R_e$ is connected to nodes \textit{emit} and \textit{0}. As shown above, the equivalent resistance is given by $R_B=\frac{R_1R_2}{R_1+R_2}$. Moreover, the equivalent voltage is given by $V_{eq}=\frac{R_2}{R_1+R_2}V_{CC}$. The current $I_E$ that goes out of the transistor's emitter is given by

\begin{equation}\label{eq:IE_GainStage}
  I_E=(1+\beta_F)I_B
\end{equation}

Where $\beta_F=178.7$ for the NPN transistor model used in Ngspice and $I_B$ is the current entering the transistor's base. Moreover, it is considered that $V_{ON}\approx0.7$ V is the voltage for the diode model, being $V_{BE_{ON}}=V_{ON}$ for the base-emitter junction. By applying KVL, the following equation can be easily obtained: $I_B=\frac{V_{eq}-V_{ON}}{R_B+(1+\beta_F)R_E}$. Moreover, we have that $I_C=\beta_FI_B$, $V_{O_1}=V_{CC}-R_CI_C$ and $V_E=R_EI_E$. Finally, having calculated these quantities, in order to make that the theoretical model considered in this laboratory assingment is valid for the chosen values for the circuit's components, it is necessary to make sure that the following relation is satisfied:

\begin{equation} \label{eq:VCE}
  V_{CE}=V_{O_1}-V_E>V_{BE_{ON}}
\end{equation}

That is, it is necessary to make sure the transistor is working in a Forward Active Region. This relation is confirmed by the values shown in Table \ref{tab:vce_vbeon}.

\begin{table}[H]
  \centering
  \begin{tabular}{|c|c|}
    \hline    
    {\bf Name} & {\bf Value [V]} \\ \hline
    \input{../mat/VCE_VBEON.tex}
  \end{tabular}
  \caption{Voltages obtained in the Gain Stage.}
  \label{tab:vce_vbeon}
\end{table}

Let us now consider the incremental analysis. The Gain Stage is, therefore, represented by the incremental circuit that learnt in class. This analysis is considered for medium frequencies, thefore the capacitor $C_b$ can be replaced by a short-circuit. Thus, the gain in the Gain Stage is given by:

\begin{equation} \label{eq:gain_GainStage}
  AV_1=(R_B||R_S)R_C\frac{R_E-g_mr_{\pi}r_o}{(r_o+R_C+R_E)(R_B||R_S+r_{\pi}+R_E)+g_mR_Er_or_{\pi}-R_E^2}
\end{equation}

Where $g_m=\frac{I_c}{V_T}$, $r_{\pi}=\frac{\beta_F}{g_m}$ and $r_o\approx\frac{V_A}{I_c}$. The thermal voltage considered was $V_T=\frac{kT}{q}$ with $T=300.15$K, $k=1.38064852\times10^{-23}$ $m^2kgs^{-2}$ and $q=1.60217662\times10^{-19}$. Moreover, the Early Voltage used in the NPN Ngspice model is $V_A=69.7$ V.
\par
This same equation can be simplified by considering $r_o=\infty$, from which $ AV_1 (r_o=\infty)=(R_B||R_S)R_C\frac{-g_mr_{\pi}}{R_B||R_S+r_{\pi}+R_E+g_mR_Er_{\pi}}$ is obtained. However, during the course of this assignment, it was verified that this approximation showed to produce worse reults, thus it wasn't considered. On ther other hand, it was considering a fairly reasonable approximation to have $R_E=0$ in equation \ref{eq:gain_GainStage}. The final results, shown in Figure \ref{fig:gain_plot} were, in fact, obtained by considering the average value of $AV_1$, obtained from equation \ref{eq:gain_GainStage} with $R_E$ given in Table \ref{tab:chosen_values}, and $AV_1$ with $R_E=0$, because the results with the average value were much more similar to those obtained in Ngspice, rather than using only one of the two. This will be further discussed in Section \ref{sec:simulation}.
\par
Finally, the input and output impedances for the Gain Stage are given respectively by:

\begin{equation} \label{eq:input_impedance_GainStage}
Z_{I1}=R_1||R_2||r_{\pi}=R_B||r_{\pi}=\frac{R_Br_{\pi}}{R_B+r_{\pi}}
\end{equation}

\begin{equation} \label{eq:output_impedance_GainStage}
  Z_{O1}=R_C||r_o=\frac{R_Cr_o}{R_c+r_o}
\end{equation}

The impedances obtained are shown in Table \ref{tab:theoretical_impedances} and will be discussed further beyond.
\par
Now, let us consider the Output Stage circuit, which is shown below.

\begin{figure}[H] \centering
  \includegraphics[width=0.8\linewidth]{CircuitDrawOutputStage.pdf}
  \caption{Output Stage circuit.}
  \label{fig:CircuitDrawOutputStage}
\end{figure}

For the Operating Point analysis, however, the capacitor is considered an open-circuit. In the incremental analysis, by considering only medium frequencies, the capacitor can also be disregarded in the model. For these reasons, the output voltage $V_{o_2}$ was drawn as the voltage drop between nodes \textit{emit2} and \textit{0}; however, in the Ngspice simulation, the output voltage will of course be considered in node \textit{out}.
\par
In the Output Stage, a PNP transistor is used. In this case, $\beta_F=227.3$ and the Early Voltage $V_A=37.2$ are the values used in Ngspice and therefore in the Octave script as well. Moreover, we have that  $V_{EB_{ON}}=V_{ON}\approx0.7$ V. For the OP analysis, firstly, we have that the input voltage is the output voltage from the Gain Stage, $V_{O_1}$. By applying KVL, the equation

\begin{equation} \label{eq:IE_OutputStage}
  I_E=\frac{V_{CC}-V_{EB_{ON}}-V_{O_1}}{R_{out}}
\end{equation}

As we can see in Table \ref{tab:theoretical_IE}, the current $I_E$ is quite bigger that the one obtained in the Gain Stage. This aspect justifies the usage of the Output Stage circuit: as seen in Table \ref{tab:theoretical_impedances}, the output impedance from the Gain Stage, $Z_{O_1}$, is quite large, thus it could not be connected to the 8 $\Omega$ resistance in the load (the audio device). Now, because of the Output Stage, part of the current $I_E$ will feed the load, which will help us at obtaining a smaller output impedance. Another thing to notice is that, as seen in Table \ref{tab:theoretical_IE}, the voltage $V_{CE}$, which is now given by $V_{CE}=-V_{O_2}-R_EI_E$, has an absolute value of 12 V, as expected, because of $V_{CC}$ and the fact that the capacitor isn't taken here into account. Therefore, this voltage is bigger than $V_{ON}$ (given by this expression because of the currents defined in the PNP transistor), which guarrantees that we are still working in a Forward Active Region, as intended.

\begin{table}[H]
  \centering
  \begin{tabular}{|c|c|}
    \hline    
    {\bf Name} & {\bf Value} \\ \hline
    \input{../mat/IE.tex}
  \end{tabular}
  \caption{Values of current $I_E$ obtained for the Gain and Output Stages by theoretical analysis.}
  \label{tab:theoretical_IE}
\end{table}

By incremental analysis, the gain and the input and output impedances of the Output Stage can be easily computed. They are given by the following equations:

\begin{equation} \label{eq:gain_OutputStage}
  AV_2=\frac{g_m}{g_{\pi}+g_E+g_o+g_m}
\end{equation}

\begin{equation} \label{eq:input_impedance_OutputStage}
  Z_{I2}=\frac{g_{\pi}+g_E+g_o+g_m}{g_{\pi}(g_{\pi}+g_E+g_o)}
\end{equation}

\begin{equation} \label{eq:output_impedance_OutputStage}
  Z_{O2}=\frac{1}{g_{\pi}+g_E+g_o+g_m}
\end{equation}

Where $g_m=\frac{I_C}{V_T}$, $g_o=\frac{I_C}{V_A}$, $g_{\pi}=\frac{g_m}{\beta_f}$ and $g_E=\frac{1}{R_{out}}$. For the Output Stage and the PNP transistor, the current $I_C$ is given by $I_C=\frac{\beta_F}{1+\beta_F}I_E$ and the values of $\beta_F$ and $V_A$ are those used for the PNP model in Ngspice, as mentioned above. Taking the whole circuit (Gain Stage+Output Stage) into account, it is now necessary to obtain the gain AV and the output impedance $Z_O$. The input impedance of the whole circuit is given by $Z_I=Z_{I1}$. By studying the circuit in an incremental analysis, the following expressions are obtained:

\begin{equation} \label{eq:gain}
  AV=\frac{\frac{1}{r_{\pi2}+Z_{O1}}+\frac{g_{m2}r_{\pi2}}{r_{\pi2}+Z_{O1}}}{\frac{1}{r_{\pi2}+Z_{O1}}+\frac{1}{R_{E2}}+\frac{1}{r_{o2}}+\frac{g_{m2}r_{\pi2}}{r_{\pi2}+Z_{O1}}}AV_1
\end{equation}

\begin{equation} \label{eq:input_impedance_OutputStage}
  Z_O=\frac{1}{g_{o2}+g_{m2}\frac{r_{\pi2}}{r_{\pi2}+Z_{O1}}+g_{e2}+\frac{1}{r_{\pi2}+Z_{O1}}}
\end{equation}

In the above expressions, the index 1 refers to values calculated for the Gain Stage and the index 2 corresponds to values determined for the Output Stage.
\par
The values obtained for all the input and output impedances were the following:

\begin{table}[H]
  \centering
  \begin{tabular}{|c|c|}
    \hline    
    {\bf Name} & {\bf Value [$\Omega$]} \\ \hline
    \input{../mat/Impedances.tex}
  \end{tabular}
  \caption{Values obtained for the impedances by theoretical analysis.}
  \label{tab:theoretical_impedances}
\end{table}

The values obtained for the gains were the following:

\begin{table}[H]
  \centering
  \begin{tabular}{|c|c|}
    \hline    
    {\bf Name} & {\bf Value} \\ \hline
    \input{../mat/Gains.tex}
  \end{tabular}
  \caption{Values obtained for the gains by theoretical analysis.}
  \label{tab:theoretical_gains}
\end{table}


The value obtained for the cut-off frequency was the following:

\begin{table}[H]
  \centering
  \begin{tabular}{|c|c|}
    \hline    
    {\bf Name} & {\bf Value [Hz]} \\ \hline
    \input{../mat/CutOff.tex}
  \end{tabular}
  \caption{Value obtained for the cut-off frequency by theoretical analysis.}
  \label{tab:theoretical_cutoff}
\end{table}

Finally, a plot with tha gain with respect to frequency is plotted below.

\begin{figure}[H] \centering
  \includegraphics[width=0.7\linewidth]{gain_plot.eps}
  \caption{Plot of the gain with respect to frequency, as obtained by theoretical analysis.}
  \label{fig:gain_plot}
\end{figure}

