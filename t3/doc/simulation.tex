\section{Simulation Analysis} \label{sec:simulation}

In order to simulate this circuit with Ngspice, the ideal transformer model was used. Because the voltage in the primary, $v_{T1}$, is known, the primary was replaced by a dependant current source and the secondary by a dependant voltage source, in order to have a voltage $v_{T2}=\frac{v_{T2}}{n}$ in the secondary. The values used for $R_1$, $R_2$, $C$, $V_{T2}$ and $n$ are those shown in Table \ref{tab:ChosenValues} of Section \ref{sec:analysis}.
\par
The plot containing the Envelope Detector's output voltage (voltage in node out1) and the Voltage Regulator's output voltage (voltage in node out2) is shown below, for 10 periods.

\begin{figure}[H]
  \begin{subfigure}{.49\linewidth}
    \centering
    \includegraphics[width=0.95\linewidth]{../sim/outsignals.pdf}
    \footnotesize
  \caption{Envelope Detector's output voltage (v(out1)) and Voltage Regulator's output voltage (v(out2)) for the first 10 periods.}
   \label{fig:out1_out2}
  \end{subfigure}
  \hspace{5mm}
  \begin{subfigure}{.49\linewidth}
    \centering
  \includegraphics[width=0.95\linewidth]{../sim/outsignals_new.pdf}
  \caption{Envelope Detector's output voltage (v(out1)) and Voltage Regulator's output voltage (v(out2)) for 10 later periods.}
  \label{fig:out1_out2_new}
  \end{subfigure}
\end{figure}

As we can see in the graphs in Figure \ref{fig:out1_out2}, the initial transitory behaviour of the voltages is still clearly present. Therefore, all the values measured in this section were only done in the time interval in which v(out1) and v(out2) are represented in Figure \ref{fig:out1_out2_new} (from 300 ms to 400 ms). Again, as was the case in Section \ref{sec:analysis}, the behaviour of these voltages is not entirely clear from these plots, because the oscillations are very small compared to the y-axis scale. Therefore, separate plots have been made for each voltage, as shown below.

\begin{figure}[H]
  \begin{subfigure}{.49\linewidth}
    \centering
    \includegraphics[width=0.95\linewidth]{../sim/out1signal.pdf}
    \footnotesize
  \caption{Envelope Detector's output voltage (v(out1)) for the time interval [300,400]ms.}
   \label{fig:out1}
  \end{subfigure}
  \hspace{5mm}
  \begin{subfigure}{.49\linewidth}
    \centering
  \includegraphics[width=0.95\linewidth]{../sim/out2signal.pdf}
  \caption{Voltage Regulator's output voltage (v(out2)) for the time interval [300,400]ms.}
  \label{fig:out2}
  \end{subfigure}
\end{figure}

In Figure \ref{fig:deviation}, the fluctuations of the Voltage Regulator's output around $12$ V (i.e., v(out2)-12) have been plotted.

\begin{figure}[H]
  \centering
  \includegraphics[width=0.5\linewidth]{../sim/deviation.pdf}
  \caption{Deviation of output signal from the target DC voltage $12$ V over the time interval [300,400]ms.}
  \label{fig:deviation}
\end{figure}

As we can see, a rather stable $12$ V voltage has been obtained.
\par
Now, as in the Theoretical Analysis, the final output's average voltage has been calculated, as well as its ripple, given by $ripple(v_{OUT})=ripple(v(out2))=max(v_{OUT})-min(v_{OUT})=max(v(out2))-min(v(out2))$. In Table \ref{tab:comparison}, the results obtained in both cases are shown side by side.

\begin{table}[H]
  \centering
  \begin{tabular}{cc}
    \begin{tabular}{|c|c|}
      \hline
      \multicolumn{2}{|c|}{\bf \color{blue} Theoretical} \\
      \hline
          {\bf Designation} & {\bf Value [V]} \\ \hline
          \input{FinalVoltage_Average_Ripple.tex}
    \end{tabular}
    \qquad
    \begin{tabular}{|c|c|}
      \hline
      \multicolumn{2}{|c|}{\bf \color{blue} Simulation} \\
      \hline    
          {\bf Designation} & {\bf Value [V]} \\ \hline
          \input{../sim/rip_tab.tex}
    \end{tabular}
  \end{tabular}
  \caption{Comparison between theoretical and simulation analysis' results.}
  \label{tab:comparison}
\end{table}


Finally, the total monetary cost and the merit $M$ of the circuit used have been calculated and are shown below in Table \ref{tab:rip}. The cost is given by $cost=cost\; of\; resistors\; +\; cost\; of\; capacitor\; +\; cost\; of\; diodes$, in which each 1$k\Omega$ in the resistances costs 1 monetary unit (MU), as well as each 1$\mu F$ in the capacitance; the cost of each diode is 0.1 MU. On the other hand, the  merit M is given by

\begin{equation} \label{eq:merit}
  M=\frac{1}{cost\times (ripple(v_{OUT})+|\overline{v}_{OUT}-12|+10^{-6})} 
\end{equation}


\begin{table}[H]
  \centering
  \begin{tabular}{|c|c|}
    \hline
        {\bf Designation} & {\bf Value [V]} \\ \hline
        \input{../sim/cost_merit_tab.tex}
  \end{tabular}
  \caption{Cost and merit obtained for this circuit.} 
  \label{tab:rip}
\end{table}

