\section{Simulation Analysis}

Simulating the ciruit indicated we have obtained the following graphic that contemplates the voltage after the envelope circuit (out1) and the voltage that actually leaves the system, supposedly 12 V.

\begin{figure}[H]
  \centering
  \small
  \includegraphics[width=0.5\linewidth]{../sim/outsignals.pdf}
  \caption{Voltage after envelope circuit(out1) and after voltage ripple circuit (out2)}
  \label{fig:out1_out2}
\end{figure}

As we can see the graph doesn't start at 0 since the system needs to stabilize so all the averages and ouputs measured were only done in the interval represented in \ref{fig:out1_out2}, this happens so we can ignore the transient voltages that appear due to the initial conditions.
In the next graph we tried observing the fluctuations of the output around 0 for that we plotted the output voltage obtained at any instant subtracted by the desired voltage 12 V.

\begin{figure}[H]
  \centering
  \includegraphics[width=0.5\linewidth]{../sim/deviation.pdf}
  \vspace{-5mm}
  \caption{Deviation of output signal from the target voltage 12 V }
  \label{fig:deviation}
\end{figure}

Doing all of the analisys we where able to obtain a stable 12V DC  with some fluctuation as seen in \ref{fig:deviation} the values obtained for the average of voltage and ripple are the following:

\begin{table}[H]
  \centering
  \begin{tabular}{|c|c|}
    \hline
        {\bf Designation} & {\bf Value [V]} \\ \hline
        \input{../sim/rip_tab.tex}
  \end{tabular}
  \caption{Voltage ripple and average out voltage} 
  \label{tab:rip}
\end{table}

Using the formula dor merit $\frac{1}{cost \times (ripple +average+10^{-6})}$ we can now obtain the cost and merit of the system.


\begin{table}[H]
  \centering
  \begin{tabular}{|c|c|}
    \hline
        {\bf Designation} & {\bf Value [V]} \\ \hline
        \input{../sim/cost_tab.tex}
  \end{tabular}
  \caption{Cost and merit} 
  \label{tab:rip}
\end{table}

