\section{Theoretical Analysis} \label{sec:analysis}

The full-wave bridge rectifier circuit, assuming ideal diodes, computes the absolute value of the sinusoidal AC voltage in the second transfomer. Therefore, the voltage in resistor $R_1$'s terminals is given by

\begin{equation} \label{eq:rectified_voltage}
  v_{O_{rectified}}(t)=|v_{t2}|
\end{equation}

Where $|v_{t2}|$ is the AC voltage in the secondary, given by $v_{t2}=\frac{v_{t1}}{n}$, assuming that the transformers' wires are coiled according to the conventional way, that is, in order to generate a downwards current through the primary and an upwards current through the secondary. The value of $n$ corresponds to the proportion of turns between the primary and the secondary, as shown in Figure \ref{fig:CircuitDraw}. Moreover, the capacitor, connected to the rectifier, transforms the rectified wave by attenuating the oscillations. Again, assuming the ideal diode model, after a time $t_{OFF}$, when the rectified signal's voltage value starts to decrease more abruptly, the current in the capacitor, given by $i_C=C\frac{dv_c}{dt}$, becomes very large, the diode eventually goes OFF and the capacitor starts decharging. This value can be computed by:

\begin{equation} \label{eq:toff}
  t_{OFF}=\frac{1}{\omega}atan\left(\frac{1}{\omega R_1C}\right)
\end{equation}

Where $\omega=2\pi f$ and $f=50$ Hz. The diode starts conducting again after $t_{ON}$, in which the rectified signal's voltage equals that of the discharging capacitor. While the capacitor is charging and the diodes are conducting, the voltage out of the envelope detector is given by $v_{O_{env}}(t)=v_{O_{rectified}}(t)$. On the other hand, while the capacitor is discharging, it is given by:

\begin{equation} \label{eq:exponential_voltage}
  v_{O_{env}}(t)=V_{t2}cos(\omega t_{OFF})e^{-\frac{t-t_{OFF}}{R_1C}}
\end{equation}

The value $V_{t2}$ is the amplitude of the voltage $v_{t2}$ in the secondary. It is worth noticing that $t_{OFF}$ in the exponential above must be summed half a period, $\frac{T}{2}=\frac{1}{2f}$, every time equation \ref{eq:exponential_voltage} is used again.

\par

Finally, the voltage regulator is used to attenuate the oscillations in the rectified signal. In this case, the ideal diode model is not used; instead, the diode model with a voltage source and a resistor is taken into account and a DC and an incremental analysis are made. The diode equation, given by

\begin{equation} \label{eq:diode_equation}
  i_D=I_S\left(e^{\frac{v_D}{\eta V_T}}-1\right)
\end{equation}

Returns the value of the current that passes in a diode. The saturation current's value used in this laboratory assignment is $I_S=1.0e10^{-14}$ A, which is the the value used by Ngspice in its default diode model. This model was utilized in Section \ref{sec:simulation}. On the other hand, the thermal voltage is given approximately by $V_T=\frac{kT}{q}\approx 25$ mV at room temperature. It was also considered that $\eta=1$, because that is the value used by Ngspice. Finally, $v_D$ is the voltage in the diode's terminals. Because there are 17 equal diodes in the voltage regulator, the voltage through each one, in terms of the DC components, is given by






\begin{table}[H]
  \centering
  \begin{tabular}{|c|c|}
    \hline    
    {\bf Name} & {\bf Value} \\ \hline
    \input{../mat/ChosenValues.tex}
  \end{tabular}
  \caption{Values used for the resistances $R_1$ and $R_2$ and for the capacitance $C$.}
  \label{tab:ChosenValues}
\end{table}




\begin{table}[H]
  \centering
  \begin{tabular}{|c|c|}
    \hline    
    {\bf Name} & {\bf Value} \\ \hline
    \input{../mat/tOFF.tex}
  \end{tabular}
  \caption{Value obtained for $t_{OFF}$.}
  \label{tab:tOFF}
\end{table}


\begin{figure}[H] \centering
  \includegraphics[width=0.7\linewidth]{envelope.eps}
  \caption{Voltages in the envelope detector and voltage regulator circuits.}
  \label{fig:envelope_regulator_voltages}
\end{figure}
